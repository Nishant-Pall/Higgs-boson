\documentclass[10pt,conference,compsocconf]{IEEEtran}

\usepackage[utf8]{inputenc} % For characters diacritic symbols
\usepackage{hyperref}
\usepackage{graphicx}	% For figure environment

\begin{document}

\title{Machine Learning Course - Class Project 1}

\author{
  Hrvoje Bušić, Dino Mujkić, Sebastijan Stevanović\\
  \textit{EPFL Lausanne, Switzerland}
}

\maketitle

\begin{abstract}
The Higgs boson is an elementary particle in the Standard Model of physics, which explains why other particles have mass. Protons during a collision at high speeds generate smaller particles as by-products of the collision, and rarely these collisions can produce a Higgs boson. As Higgs boson decays rapidly into other particles, scientists don't observe it directly, but rather measure its decay signature. Many decay signatures look similar, and in this project we used different machine learning algorithms to train models for predicting whether the given event's signature was the result of a Higgs boson (signal) or some other process/particle (background).
\end{abstract}



\end{document}